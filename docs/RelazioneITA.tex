% citazioni dall bibliografia:
% \textsuperscript{\cite{cas}}
%
% codice con linguaggio + caption:
% \begin{lstlisting}[language=xml, caption=Richiesta per un Access Token]code\end{lstlisting}
%
% immagine + caption + label (utile per \ref{valore_label}
%\begin{figure}[H]
%\begin{center}
%    \includegraphics [width=4in]{federa_sequenza.eps}
%\caption{Diagramma di sequeza per Autenticazione Federata}
%\label{autenticazioneFederata}
%\end{center}
%\end{figure}

\documentclass{article}
\usepackage{graphicx}
\usepackage{epstopdf}
\usepackage{wrapfig}
\usepackage{color}
\usepackage[utf8x]{inputenc}
\usepackage[T1]{fontenc}
\usepackage{enumitem}
\usepackage{listings}
\usepackage{color}
\usepackage[a4paper]{geometry}
\usepackage{amsmath}
\usepackage{amssymb}
\usepackage{mathpazo,graphics,graphicx,amsopn,url, epsf}
% Include the helvet package (helvetica font)
\usepackage[scaled]{helvet}
\usepackage[toc,page]{appendix}
\usepackage{float}
\usepackage[font={scriptsize,it}]{caption}

% Set the default font to be sans-seriF
\renewcommand*{\familydefault}{\sfdefault}

% Change 'table of contents' -> contents -> to italian 'Indice'
\renewcommand{\contentsname}{Indice}

% Change the 'figure' caption to the italian 'figura'
\renewcommand{\figurename}{Figura}

% Add a caption to lstlisting (code)
\renewcommand{\lstlistingname}{Codice}

% Renename References to Bibliografia
\renewcommand{\refname}{Sitografia}

\definecolor{dkgreen}{rgb}{0,0.6,0}
\definecolor{mauve}{rgb}{0.58,0,0.82}
\definecolor{lightgray}{gray}{0.5}

\lstset{frame=tb,
    aboveskip=3mm,
    belowskip=3mm,
    showstringspaces=false,
    columns=flexible,
    basicstyle={\small\ttfamily},
    numbers=none,
    numberstyle=\tiny\color{lightgray},
    keywordstyle=\color{blue},
    commentstyle=\color{dkgreen},
    stringstyle=\color{mauve},
    breaklines=true,
    breakatwhitespace=true,
    captionpos=b,
    tabsize=2,
}

\lstset{literate=
    {á}{{\'a}}1 {é}{{\'e}}1 {í}{{\'i}}1 {ó}{{\'o}}1 {ú}{{\'u}}1
    {Á}{{\'A}}1 {É}{{\'E}}1 {Í}{{\'I}}1 {Ó}{{\'O}}1 {Ú}{{\'U}}1
    {à}{{\`a}}1 {è}{{\'e}}1 {ì}{{\`i}}1 {ò}{{\`o}}1 {ù}{{\`u}}1
    {À}{{\`A}}1 {È}{{\'E}}1 {Ì}{{\`I}}1 {Ò}{{\`O}}1 {Ù}{{\`U}}1
    {ä}{{\"a}}1 {ë}{{\"e}}1 {ï}{{\"i}}1 {ö}{{\"o}}1 {ü}{{\"u}}1
    {Ä}{{\"A}}1 {Ë}{{\"E}}1 {Ï}{{\"I}}1 {Ö}{{\"O}}1 {Ü}{{\"U}}1
    {â}{{\^a}}1 {ê}{{\^e}}1 {î}{{\^i}}1 {ô}{{\^o}}1 {û}{{\^u}}1
    {Â}{{\^A}}1 {Ê}{{\^E}}1 {Î}{{\^I}}1 {Ô}{{\^O}}1 {Û}{{\^U}}1
    {œ}{{\oe}}1 {Œ}{{\OE}}1 {æ}{{\ae}}1 {Æ}{{\AE}}1 {ß}{{\ss}}1
    {ç}{{\c c}}1 {Ç}{{\c C}}1 {ø}{{\o}}1 {å}{{\r a}}1 {Å}{{\r A}}1
    {€}{{\EUR}}1 {£}{{\pounds}}1
}

\sloppy
\setlength{\parindent}{0pt}
\setlist[itemize]{itemsep=1pt}

\begin{document}

\begin{titlepage}
    \begin{center}
        {{\large{\textsc{ALMA MATER STUDIORUM - UNIVERSITÀ DI BOLOGNA}}}}
        \rule[0.1cm]{14cm}{0.1mm}
        \rule[0.5cm]{14cm}{0.6mm}
        {\textsc { SCUOLA DI INGEGNERIA E ARCHITETTURA } }\\
        {\small{\textsc { Corso di laurea in ingegneria informatica}}}
    \end{center}
    \vspace*{\fill}
    \begin{center}
        {\Large\textbf{Face Miner\\[1\baselineskip]
        Un approccio di Data Mining al problema della Face Detection}} \\
    \end{center}
    \vspace*{\fill}
    \par
    \noindent
    \begin{minipage}[t]{0.47\textwidth}
        {\large{Relazione di:}\\[1\baselineskip]
        {\bf Paolo Galeone}}
    \end{minipage}
    \hfill
%\begin{minipage}[t]{0.47\textwidth}\raggedleft
%    {\large{Relatore:}\\[1\baselineskip]
%    {\bf Prof. Paolo Ciaccia}}
%\end{minipage}
    \vspace{20mm}
    \begin{center}
        \rule[0.1cm]{14cm}{0.1mm}
        \rule[0.5cm]{14cm}{0.6mm}
        Anno Accademico 2015/2016\\
%    Sessione II
    \end{center}
\end{titlepage}
\clearpage
\tableofcontents
\clearpage

\section{Introduzione}

TODO: ultima sezione da scrivere, in cui viene spiegato brevemente in cosa consiste il lavoro. Su cosa si basa, i risultati e le modifiche.
\clearpage
\section{Face Detection}
La Face Detection (Riconoscimento facciale) è un caso specifico del problema di object-class detection.

Il compito della object-class detection è quello di trovare la posizione e la dimensione di tutti gli oggetti che appartengono ad una determinata classe. Nel caso della face detection, la classe è quella dei volti.

Un essere umano può effettuare il task di face detection in maniera naturale a differenza di una macchina che necessità di precise informazioni e vincoli da rispettare.

Tra i vari algoritmi presentati in letteratura, il più noto ed utilizzato algoritmo è il Viola-Jones framework.

\subsection{Viola-Jones framework}
L'algoritmo di Viola-Jones è il primo algoritmo con performance tali da poter essere utilizzato in applicazioni di face detectoin in real time.

È un algoritmo basato sull'estrazione di determinate caratteristiche (le HAAR like features) e l'uso di tecniche di apprendimento automatico.

Il framework è organizzato come segue:
\begin{enumerate}
 \item \textbf{Features extractor }
\end{enumerate}




\clearpage

\clearpage
\begin{thebibliography}{9}
    \bibitem{ref1}
        \emph{Sensivity and Specificity}
        https://en.wikipedia.org/wiki/Sensitivity\_and\_specificity

    \bibitem{ref2}
        \emph{Precision and Recall}
        https://en.wikipedia.org/wiki/Precision\_and\_recall

\end{thebibliography}

\end{document}